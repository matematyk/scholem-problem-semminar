\begin{frame}{Ideały pierwsze}
$I \subset O$ jest idałem 
$a, b \in I$

Ideał pierwszy: $(2,x) \subset \mathbb {Z}[x]$
Ideał nie jest pierwszy  $(2,x^{2}+5)\subset \mathbb {Z}[x]$
\begin{przyklad}
$$
    {\displaystyle (2,x^{2}+5)\subset \mathbb {Z}[x]}
$$
$$ 
    {\displaystyle x^{2}+5-2\cdot 3=(x-1)(x+1)\in (2,x^{2}+5)}
$$ jednak $(x-1) \notin (2,x^{2} + 5) $
\end{przyklad}

\end{frame}

%\begin{frame}{Przykład:}
%\begin{itemize}
%   \item  Rozważmy ciąg: 
%   \item  $$ 0, 0, 1, 0, 1, 0, 2, 0, 3, 0, 5, 0, 8, 0, ...$$
%    \item      Zauważmy szereg Fibonacciego pomiędzy zerami. 
%    \item  Zdefiniowany takim wzorem:
%    \item  $F(i) = F(i-2) + F(i-4)$
%    \item  Startując od przypadku podstawowego $F(1) = F(2) = F(4) = 0$i $F(3) = 1$
%    \item  Wówczas $F(i) = 0 \iff \text{i jest 1 lub parzyste} $  
%    \item pozycje na którym mamy zero możemy podzielić na singleton ${1}$ zbiór skończony i ciąg pełnie arytmetyczny ciąg parzystych liczb. 
%\end{itemize}
%\end{frame}


%\begin{frame}{Skolem problem}
%    Czy dane liniowe rekurencyjne ma zero. 
%    $u_n = 0$ dla pewnego $n$?
%    \begin{itemize}
%        \item Istnieje algorytm pozwalający sprawdzić, czy istnieje nieskończona ilość zer, a jeśli tak, to znaleźć rozkład tych zer na zbiory okresowe, gwarantowane przez twierdzenie Skolema-Mahlera-Lecha. 
%        \item Nie wiadomo jednak, czy istnieje algorytm pozwalający określić, czy sekwencja rekurencyjna ma jakieś nieperidoczyne zera. 
%    \end{itemize}
%\end{frame}